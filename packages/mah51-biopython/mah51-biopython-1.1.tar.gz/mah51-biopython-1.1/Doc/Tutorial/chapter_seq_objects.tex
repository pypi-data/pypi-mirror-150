\chapter{Sequence objects}
\label{chapter:seq_objects}

Biological sequences are arguably the central object in Bioinformatics, and in this chapter we'll introduce the Biopython mechanism for dealing with sequences, the \verb|Seq| object.
Chapter~\ref{chapter:seq_annot} will introduce the related \verb|SeqRecord| object, which combines the sequence information with any annotation, used again in Chapter~\ref{chapter:seqio} for Sequence Input/Output.

Sequences are essentially strings of letters like \verb|AGTACACTGGT|, which seems very natural since this is the most common way that sequences are seen in biological file formats.

The most important difference between \verb|Seq| objects and standard Python strings is they have different methods.
Although the \verb|Seq| object supports many of the same methods as a plain string, its \verb|translate()| method differs by doing biological translation, and there are also additional biologically relevant methods like \verb|reverse_complement()|.

\section{Sequences act like strings}

In most ways, we can deal with Seq objects as if they were normal Python strings, for example getting the length, or iterating over the elements:

%doctest
\begin{minted}{pycon}
>>> from Bio.Seq import Seq
>>> my_seq = Seq("GATCG")
>>> for index, letter in enumerate(my_seq):
...     print("%i %s" % (index, letter))
...
0 G
1 A
2 T
3 C
4 G
>>> print(len(my_seq))
5
\end{minted}

You can access elements of the sequence in the same way as for strings (but remember, Python counts from zero!):

%cont-doctest
\begin{minted}{pycon}
>>> print(my_seq[0])  # first letter
G
>>> print(my_seq[2])  # third letter
T
>>> print(my_seq[-1])  # last letter
G
\end{minted}

The \verb|Seq| object has a \verb|.count()| method, just like a string.
Note that this means that like a Python string, this gives a
\emph{non-overlapping} count:

%doctest
\begin{minted}{pycon}
>>> from Bio.Seq import Seq
>>> "AAAA".count("AA")
2
>>> Seq("AAAA").count("AA")
2
\end{minted}

\noindent For some biological uses, you may actually want an overlapping count
(i.e. $3$ in this trivial example). When searching for single letters, this
makes no difference:

%doctest
\begin{minted}{pycon}
>>> from Bio.Seq import Seq
>>> my_seq = Seq("GATCGATGGGCCTATATAGGATCGAAAATCGC")
>>> len(my_seq)
32
>>> my_seq.count("G")
9
>>> 100 * float(my_seq.count("G") + my_seq.count("C")) / len(my_seq)
46.875
\end{minted}

While you could use the above snippet of code to calculate a GC\%, note that  the \verb|Bio.SeqUtils| module has several GC functions already built.  For example:

%doctest
\begin{minted}{pycon}
>>> from Bio.Seq import Seq
>>> from Bio.SeqUtils import GC
>>> my_seq = Seq("GATCGATGGGCCTATATAGGATCGAAAATCGC")
>>> GC(my_seq)
46.875
\end{minted}

\noindent Note that using the \verb|Bio.SeqUtils.GC()| function should automatically cope with mixed case sequences and the ambiguous nucleotide S which means G or C.

Also note that just like a normal Python string, the \verb|Seq| object is in some ways ``read-only''.  If you need to edit your sequence, for example simulating a point mutation, look at the Section~\ref{sec:mutable-seq} below which talks about the \verb|MutableSeq| object.

\section{Slicing a sequence}

A more complicated example, let's get a slice of the sequence:

%doctest
\begin{minted}{pycon}
>>> from Bio.Seq import Seq
>>> my_seq = Seq("GATCGATGGGCCTATATAGGATCGAAAATCGC")
>>> my_seq[4:12]
Seq('GATGGGCC')
\end{minted}

Note that `Seq` objects follow the usual indexing conventions for Python strings, with the first element of the sequence numbered 0. When you do a slice the first item is included (i.e.~4 in this case) and the last is excluded (12 in this case).

Also like a Python string, you can do slices with a start, stop and \emph{stride} (the step size, which defaults to one).  For example, we can get the first, second and third codon positions of this DNA sequence:

%cont-doctest
\begin{minted}{pycon}
>>> my_seq[0::3]
Seq('GCTGTAGTAAG')
>>> my_seq[1::3]
Seq('AGGCATGCATC')
>>> my_seq[2::3]
Seq('TAGCTAAGAC')
\end{minted}

Another stride trick you might have seen with a Python string is the use of a -1 stride to reverse the string.  You can do this with a \verb|Seq| object too:

%cont-doctest
\begin{minted}{pycon}
>>> my_seq[::-1]
Seq('CGCTAAAAGCTAGGATATATCCGGGTAGCTAG')
\end{minted}

\section{Turning Seq objects into strings}
\label{sec:seq-to-string}

If you really do just need a plain string, for example to write to a file, or insert into a database, then this is very easy to get:

%cont-doctest
\begin{minted}{pycon}
>>> str(my_seq)
'GATCGATGGGCCTATATAGGATCGAAAATCGC'
\end{minted}

Since calling \verb|str()| on a \verb|Seq| object returns the full sequence as a string,
you often don't actually have to do this conversion explicitly.
Python does this automatically in the print function:

%cont-doctest
\begin{minted}{pycon}
>>> print(my_seq)
GATCGATGGGCCTATATAGGATCGAAAATCGC
\end{minted}

You can also use the \verb|Seq| object directly with a \verb|%s| placeholder when using the Python string formatting or interpolation operator (\verb|%|):

%cont-doctest
\begin{minted}{pycon}
>>> fasta_format_string = ">Name\n%s\n" % my_seq
>>> print(fasta_format_string)
>Name
GATCGATGGGCCTATATAGGATCGAAAATCGC
<BLANKLINE>
\end{minted}

\noindent This line of code constructs a simple FASTA format record (without worrying about line wrapping).
Section~\ref{sec:SeqRecord-format} describes a neat way to get a FASTA formatted
string from a \verb|SeqRecord| object, while the more general topic of reading and
writing FASTA format sequence files is covered in Chapter~\ref{chapter:seqio}.

\section{Concatenating or adding sequences}

Two \verb|Seq| objects can be concatenated by adding them:

%doctest
\begin{minted}{pycon}
>>> from Bio.Seq import Seq
>>> seq1 = Seq("ACGT")
>>> seq2 = Seq("AACCGG")
>>> seq1 + seq2
Seq('ACGTAACCGG')
\end{minted}

Biopython does not check the sequence contents and will not raise an exception if for example you concatenate a protein sequence and a DNA sequence (which is likely a mistake):

%doctest
\begin{minted}{pycon}
>>> from Bio.Seq import Seq
>>> protein_seq = Seq("EVRNAK")
>>> dna_seq = Seq("ACGT")
>>> protein_seq + dna_seq
Seq('EVRNAKACGT')
\end{minted}

You may often have many sequences to add together, which can be done with a for loop like this:

%doctest
\begin{minted}{pycon}
>>> from Bio.Seq import Seq
>>> list_of_seqs = [Seq("ACGT"), Seq("AACC"), Seq("GGTT")]
>>> concatenated = Seq("")
>>> for s in list_of_seqs:
...     concatenated += s
...
>>> concatenated
Seq('ACGTAACCGGTT')
\end{minted}

Like Python strings, Biopython \verb|Seq| also has a \verb|.join| method:

%doctest
\begin{minted}{pycon}
>>> from Bio.Seq import Seq
>>> contigs = [Seq("ATG"), Seq("ATCCCG"), Seq("TTGCA")]
>>> spacer = Seq("N" * 10)
>>> spacer.join(contigs)
Seq('ATGNNNNNNNNNNATCCCGNNNNNNNNNNTTGCA')
\end{minted}

\section{Changing case}

Python strings have very useful \verb|upper| and \verb|lower| methods for changing the case.
For example,

%doctest
\begin{minted}{pycon}
>>> from Bio.Seq import Seq
>>> dna_seq = Seq("acgtACGT")
>>> dna_seq
Seq('acgtACGT')
>>> dna_seq.upper()
Seq('ACGTACGT')
>>> dna_seq.lower()
Seq('acgtacgt')
\end{minted}

These are useful for doing case insensitive matching:

%cont-doctest
\begin{minted}{pycon}
>>> "GTAC" in dna_seq
False
>>> "GTAC" in dna_seq.upper()
True
\end{minted}

\section{Nucleotide sequences and (reverse) complements}
\label{sec:seq-reverse-complement}

For nucleotide sequences, you can easily obtain the complement or reverse
complement of a \verb|Seq| object using its built-in methods:

%doctest
\begin{minted}{pycon}
>>> from Bio.Seq import Seq
>>> my_seq = Seq("GATCGATGGGCCTATATAGGATCGAAAATCGC")
>>> my_seq
Seq('GATCGATGGGCCTATATAGGATCGAAAATCGC')
>>> my_seq.complement()
Seq('CTAGCTACCCGGATATATCCTAGCTTTTAGCG')
>>> my_seq.reverse_complement()
Seq('GCGATTTTCGATCCTATATAGGCCCATCGATC')
\end{minted}

As mentioned earlier, an easy way to just reverse a \verb|Seq| object (or a
Python string) is slice it with -1 step:

%cont-doctest
\begin{minted}{pycon}
>>> my_seq[::-1]
Seq('CGCTAAAAGCTAGGATATATCCGGGTAGCTAG')
\end{minted}

If you do accidentally end up trying to do something weird like taking the
(reverse) complement of a protein sequence, the results are biologically
meaningless:

%doctest
\begin{minted}{pycon}
>>> from Bio.Seq import Seq
>>> protein_seq = Seq("EVRNAK")
>>> protein_seq.complement()
Seq('EBYNTM')
\end{minted}

Here the letter ``E'' is not a valid IUPAC ambiguity code for nucleotides,
so was not complemented. However, ``V'' means ``A'', ``C'' or ``G'' and
has complement ``B``, and so on.

The example in Section~\ref{sec:SeqIO-reverse-complement} combines the \verb|Seq|
object's reverse complement method with \verb|Bio.SeqIO| for sequence input/output.

\section{Transcription}
Before talking about transcription, I want to try to clarify the strand issue.
Consider the following (made up) stretch of double stranded DNA which
encodes a short peptide:

\begin{tabular}{rcl}
\\
   & {\small DNA coding strand (aka Crick strand, strand $+1$)} & \\
5' & \texttt{ATGGCCATTGTAATGGGCCGCTGAAAGGGTGCCCGATAG} & 3' \\
   & \texttt{|||||||||||||||||||||||||||||||||||||||} & \\
3' & \texttt{TACCGGTAACATTACCCGGCGACTTTCCCACGGGCTATC} & 5' \\
   & {\small DNA template strand (aka Watson strand, strand $-1$)} & \\
\\
   & {\LARGE $|$} &\\
   & Transcription & \\
   & {\LARGE $\downarrow$} &\\
\\
5' & \texttt{AUGGCCAUUGUAAUGGGCCGCUGAAAGGGUGCCCGAUAG} & 3' \\
   & {\small Single stranded messenger RNA} & \\
\\
\end{tabular}

The actual biological transcription process works from the template strand, doing a reverse complement (TCAG $\rightarrow$ CUGA) to give the mRNA.  However, in Biopython and bioinformatics in general, we typically work directly with the coding strand because this means we can get the mRNA sequence just by switching T $\rightarrow$ U.

Now let's actually get down to doing a transcription in Biopython.  First, let's create \verb|Seq| objects for the coding and template DNA strands:

%doctest
\begin{minted}{pycon}
>>> from Bio.Seq import Seq
>>> coding_dna = Seq("ATGGCCATTGTAATGGGCCGCTGAAAGGGTGCCCGATAG")
>>> coding_dna
Seq('ATGGCCATTGTAATGGGCCGCTGAAAGGGTGCCCGATAG')
>>> template_dna = coding_dna.reverse_complement()
>>> template_dna
Seq('CTATCGGGCACCCTTTCAGCGGCCCATTACAATGGCCAT')
\end{minted}
\noindent These should match the figure above - remember by convention nucleotide sequences are normally read from the 5' to 3' direction, while in the figure the template strand is shown reversed.

Now let's transcribe the coding strand into the corresponding mRNA, using the \verb|Seq| object's built in \verb|transcribe| method:

%cont-doctest
\begin{minted}{pycon}
>>> coding_dna
Seq('ATGGCCATTGTAATGGGCCGCTGAAAGGGTGCCCGATAG')
>>> messenger_rna = coding_dna.transcribe()
>>> messenger_rna
Seq('AUGGCCAUUGUAAUGGGCCGCUGAAAGGGUGCCCGAUAG')
\end{minted}
\noindent As you can see, all this does is to replace T by U.

If you do want to do a true biological transcription starting with the template strand, then this becomes a two-step process:

%cont-doctest
\begin{minted}{pycon}
>>> template_dna.reverse_complement().transcribe()
Seq('AUGGCCAUUGUAAUGGGCCGCUGAAAGGGUGCCCGAUAG')
\end{minted}

The \verb|Seq| object also includes a back-transcription method for going from the mRNA to the coding strand of the DNA.  Again, this is a simple U $\rightarrow$ T substitution:

%doctest
\begin{minted}{pycon}
>>> from Bio.Seq import Seq
>>> messenger_rna = Seq("AUGGCCAUUGUAAUGGGCCGCUGAAAGGGUGCCCGAUAG")
>>> messenger_rna
Seq('AUGGCCAUUGUAAUGGGCCGCUGAAAGGGUGCCCGAUAG')
>>> messenger_rna.back_transcribe()
Seq('ATGGCCATTGTAATGGGCCGCTGAAAGGGTGCCCGATAG')
\end{minted}

\emph{Note:} The \verb|Seq| object's \verb|transcribe| and \verb|back_transcribe| methods
were added in Biopython 1.49.  For older releases you would have to use the \verb|Bio.Seq|
module's functions instead, see Section~\ref{sec:seq-module-functions}.

\section{Translation}
\label{sec:translation}
Sticking with the same example discussed in the transcription section above,
now let's translate this mRNA into the corresponding protein sequence - again taking
advantage of one of the \verb|Seq| object's biological methods:

%doctest
\begin{minted}{pycon}
>>> from Bio.Seq import Seq
>>> messenger_rna = Seq("AUGGCCAUUGUAAUGGGCCGCUGAAAGGGUGCCCGAUAG")
>>> messenger_rna
Seq('AUGGCCAUUGUAAUGGGCCGCUGAAAGGGUGCCCGAUAG')
>>> messenger_rna.translate()
Seq('MAIVMGR*KGAR*')
\end{minted}

You can also translate directly from the coding strand DNA sequence:

%doctest
\begin{minted}{pycon}
>>> from Bio.Seq import Seq
>>> coding_dna = Seq("ATGGCCATTGTAATGGGCCGCTGAAAGGGTGCCCGATAG")
>>> coding_dna
Seq('ATGGCCATTGTAATGGGCCGCTGAAAGGGTGCCCGATAG')
>>> coding_dna.translate()
Seq('MAIVMGR*KGAR*')
\end{minted}

You should notice in the above protein sequences that in addition to the end stop character, there is an internal stop as well.  This was a deliberate choice of example, as it gives an excuse to talk about some optional arguments, including different translation tables (Genetic Codes).

The translation tables available in Biopython are based on those \href{https://www.ncbi.nlm.nih.gov/Taxonomy/Utils/wprintgc.cgi}{from the NCBI} (see the next section of this tutorial).  By default, translation will use the \emph{standard} genetic code (NCBI table id 1).
Suppose we are dealing with a mitochondrial sequence.  We need to tell the translation function to use the relevant genetic code instead:

%cont-doctest
\begin{minted}{pycon}
>>> coding_dna.translate(table="Vertebrate Mitochondrial")
Seq('MAIVMGRWKGAR*')
\end{minted}

You can also specify the table using the NCBI table number which is shorter, and often included in the feature annotation of GenBank files:

%cont-doctest
\begin{minted}{pycon}
>>> coding_dna.translate(table=2)
Seq('MAIVMGRWKGAR*')
\end{minted}

Now, you may want to translate the nucleotides up to the first in frame stop codon,
and then stop (as happens in nature):

%cont-doctest
\begin{minted}{pycon}
>>> coding_dna.translate()
Seq('MAIVMGR*KGAR*')
>>> coding_dna.translate(to_stop=True)
Seq('MAIVMGR')
>>> coding_dna.translate(table=2)
Seq('MAIVMGRWKGAR*')
>>> coding_dna.translate(table=2, to_stop=True)
Seq('MAIVMGRWKGAR')
\end{minted}
\noindent Notice that when you use the \verb|to_stop| argument, the stop codon itself
is not translated - and the stop symbol is not included at the end of your protein
sequence.

You can even specify the stop symbol if you don't like the default asterisk:

%cont-doctest
\begin{minted}{pycon}
>>> coding_dna.translate(table=2, stop_symbol="@")
Seq('MAIVMGRWKGAR@')
\end{minted}

Now, suppose you have a complete coding sequence CDS, which is to say a
nucleotide sequence (e.g. mRNA -- after any splicing) which is a whole number
of codons (i.e. the length is a multiple of three), commences with a start
codon, ends with a stop codon, and has no internal in-frame stop codons.
In general, given a complete CDS, the default translate method will do what
you want (perhaps with the \verb|to_stop| option). However, what if your
sequence uses a non-standard start codon? This happens a lot in bacteria --
for example the gene yaaX in \texttt{E. coli} K12:

%TODO - handle line wrapping in doctest?
\begin{minted}{pycon}
>>> from Bio.Seq import Seq
>>> gene = Seq(
...     "GTGAAAAAGATGCAATCTATCGTACTCGCACTTTCCCTGGTTCTGGTCGCTCCCATGGCA"
...     "GCACAGGCTGCGGAAATTACGTTAGTCCCGTCAGTAAAATTACAGATAGGCGATCGTGAT"
...     "AATCGTGGCTATTACTGGGATGGAGGTCACTGGCGCGACCACGGCTGGTGGAAACAACAT"
...     "TATGAATGGCGAGGCAATCGCTGGCACCTACACGGACCGCCGCCACCGCCGCGCCACCAT"
...     "AAGAAAGCTCCTCATGATCATCACGGCGGTCATGGTCCAGGCAAACATCACCGCTAA"
... )
>>> gene.translate(table="Bacterial")
Seq('VKKMQSIVLALSLVLVAPMAAQAAEITLVPSVKLQIGDRDNRGYYWDGGHWRDH...HR*',
ProteinAlpabet())
>>> gene.translate(table="Bacterial", to_stop=True)
Seq('VKKMQSIVLALSLVLVAPMAAQAAEITLVPSVKLQIGDRDNRGYYWDGGHWRDH...HHR')
\end{minted}

\noindent In the bacterial genetic code \texttt{GTG} is a valid start codon,
and while it does \emph{normally} encode Valine, if used as a start codon it
should be translated as methionine. This happens if you tell Biopython your
sequence is a complete CDS:

%TODO - handle line wrapping in doctest?
\begin{minted}{pycon}
>>> gene.translate(table="Bacterial", cds=True)
Seq('MKKMQSIVLALSLVLVAPMAAQAAEITLVPSVKLQIGDRDNRGYYWDGGHWRDH...HHR')
\end{minted}

In addition to telling Biopython to translate an alternative start codon as
methionine, using this option also makes sure your sequence really is a valid
CDS (you'll get an exception if not).

The example in Section~\ref{sec:SeqIO-translate} combines the \verb|Seq| object's
translate method with \verb|Bio.SeqIO| for sequence input/output.

\section{Translation Tables}

In the previous sections we talked about the \verb|Seq| object translation method (and mentioned the equivalent function in the \verb|Bio.Seq| module -- see
Section~\ref{sec:seq-module-functions}).
Internally these use codon table objects derived from the NCBI information at
\url{ftp://ftp.ncbi.nlm.nih.gov/entrez/misc/data/gc.prt}, also shown on
\url{https://www.ncbi.nlm.nih.gov/Taxonomy/Utils/wprintgc.cgi} in a much more readable layout.

As before, let's just focus on two choices: the Standard translation table, and the
translation table for Vertebrate Mitochondrial DNA.

%doctest
\begin{minted}{pycon}
>>> from Bio.Data import CodonTable
>>> standard_table = CodonTable.unambiguous_dna_by_name["Standard"]
>>> mito_table = CodonTable.unambiguous_dna_by_name["Vertebrate Mitochondrial"]
\end{minted}

Alternatively, these tables are labeled with ID numbers 1 and 2, respectively:

%cont-doctest
\begin{minted}{pycon}
>>> from Bio.Data import CodonTable
>>> standard_table = CodonTable.unambiguous_dna_by_id[1]
>>> mito_table = CodonTable.unambiguous_dna_by_id[2]
\end{minted}

You can compare the actual tables visually by printing them:
%TODO - handle <BLANKLINE> automatically in doctest?
\begin{minted}{pycon}
>>> print(standard_table)
Table 1 Standard, SGC0

  |  T      |  C      |  A      |  G      |
--+---------+---------+---------+---------+--
T | TTT F   | TCT S   | TAT Y   | TGT C   | T
T | TTC F   | TCC S   | TAC Y   | TGC C   | C
T | TTA L   | TCA S   | TAA Stop| TGA Stop| A
T | TTG L(s)| TCG S   | TAG Stop| TGG W   | G
--+---------+---------+---------+---------+--
C | CTT L   | CCT P   | CAT H   | CGT R   | T
C | CTC L   | CCC P   | CAC H   | CGC R   | C
C | CTA L   | CCA P   | CAA Q   | CGA R   | A
C | CTG L(s)| CCG P   | CAG Q   | CGG R   | G
--+---------+---------+---------+---------+--
A | ATT I   | ACT T   | AAT N   | AGT S   | T
A | ATC I   | ACC T   | AAC N   | AGC S   | C
A | ATA I   | ACA T   | AAA K   | AGA R   | A
A | ATG M(s)| ACG T   | AAG K   | AGG R   | G
--+---------+---------+---------+---------+--
G | GTT V   | GCT A   | GAT D   | GGT G   | T
G | GTC V   | GCC A   | GAC D   | GGC G   | C
G | GTA V   | GCA A   | GAA E   | GGA G   | A
G | GTG V   | GCG A   | GAG E   | GGG G   | G
--+---------+---------+---------+---------+--
\end{minted}
\noindent and:
\begin{minted}{pycon}
>>> print(mito_table)
Table 2 Vertebrate Mitochondrial, SGC1

  |  T      |  C      |  A      |  G      |
--+---------+---------+---------+---------+--
T | TTT F   | TCT S   | TAT Y   | TGT C   | T
T | TTC F   | TCC S   | TAC Y   | TGC C   | C
T | TTA L   | TCA S   | TAA Stop| TGA W   | A
T | TTG L   | TCG S   | TAG Stop| TGG W   | G
--+---------+---------+---------+---------+--
C | CTT L   | CCT P   | CAT H   | CGT R   | T
C | CTC L   | CCC P   | CAC H   | CGC R   | C
C | CTA L   | CCA P   | CAA Q   | CGA R   | A
C | CTG L   | CCG P   | CAG Q   | CGG R   | G
--+---------+---------+---------+---------+--
A | ATT I(s)| ACT T   | AAT N   | AGT S   | T
A | ATC I(s)| ACC T   | AAC N   | AGC S   | C
A | ATA M(s)| ACA T   | AAA K   | AGA Stop| A
A | ATG M(s)| ACG T   | AAG K   | AGG Stop| G
--+---------+---------+---------+---------+--
G | GTT V   | GCT A   | GAT D   | GGT G   | T
G | GTC V   | GCC A   | GAC D   | GGC G   | C
G | GTA V   | GCA A   | GAA E   | GGA G   | A
G | GTG V(s)| GCG A   | GAG E   | GGG G   | G
--+---------+---------+---------+---------+--
\end{minted}

You may find these following properties useful -- for example if you are trying
to do your own gene finding:

%cont-doctest
\begin{minted}{pycon}
>>> mito_table.stop_codons
['TAA', 'TAG', 'AGA', 'AGG']
>>> mito_table.start_codons
['ATT', 'ATC', 'ATA', 'ATG', 'GTG']
>>> mito_table.forward_table["ACG"]
'T'
\end{minted}

\section{Comparing Seq objects}
\label{sec:seq-comparison}

Sequence comparison is actually a very complicated topic, and there is no easy
way to decide if two sequences are equal. The basic problem is the meaning of
the letters in a sequence are context dependent - the letter ``A'' could be part
of a DNA, RNA or protein sequence. Biopython can track the molecule type, so
comparing two \verb|Seq| objects could mean considering this too.

Should a DNA fragment ``ACG'' and an RNA fragment ``ACG'' be equal? What about
the peptide ``ACG``? Or the Python string ``ACG``?
In everyday use, your sequences will generally all be the same type of
(all DNA, all RNA, or all protein).
Well, as of Biopython 1.65, sequence comparison only looks at the sequence
and compares like the Python string.

%doctest
\begin{minted}{pycon}
>>> from Bio.Seq import Seq
>>> seq1 = Seq("ACGT")
>>> "ACGT" == seq1
True
>>> seq1 == "ACGT"
True
\end{minted}

As an extension to this, using sequence objects as keys in a Python dictionary
is equivalent to using the sequence as a plain string for the key.
See also Section~\ref{sec:seq-to-string}.

\section{Sequences with unknown sequence contents}

In some cases, the length of a sequence may be known but not the actual letters constituting it. For example, GenBank and EMBL files may represent a genomic DNA sequence only by its config information, without specifying the sequence contents explicitly. Such sequences can be represented by creating a \verb|Seq| object with the argument \verb|None|, followed by the sequence length:

%doctest
\begin{minted}{pycon}
>>> from Bio.Seq import Seq
>>> unknown_seq = Seq(None, 10)
\end{minted}

The \verb|Seq| object thus created has a well-defined length. Any attempt to access the sequence contents, however, will raise an \verb|UndefinedSequenceError|:

%cont-doctest
\begin{minted}{pycon}
>>> unknown_seq
Seq(None, length=10)
>>> len(unknown_seq)
10
>>> print(unknown_seq)
Traceback (most recent call last):
...
Bio.Seq.UndefinedSequenceError: Sequence content is undefined
>>>
\end{minted}

\section{Sequences with partially defined sequence contents}

Sometimes the sequence contents is defined for parts of the sequence only, and undefined elsewhere. For example, the following excerpt of a MAF (Multiple Alignment Format) file shows an alignment of human, chimp, macaque, mouse, rat, dog, and opossum  genome sequences:

\begin{minted}{text}
s hg38.chr7     117512683 36 + 159345973 TTGAAAACCTGAATGTGAGAGTCAGTCAAGGATAGT
s panTro4.chr7  119000876 36 + 161824586 TTGAAAACCTGAATGTGAGAGTCACTCAAGGATAGT
s rheMac3.chr3  156330991 36 + 198365852 CTGAAATCCTGAATGTGAGAGTCAATCAAGGATGGT
s mm10.chr6      18207101 36 + 149736546 CTGAAAACCTAAGTAGGAGAATCAACTAAGGATAAT
s rn5.chr4       42326848 36 + 248343840 CTGAAAACCTAAGTAGGAGAGACAGTTAAAGATAAT
s canFam3.chr14  56325207 36 +  60966679 TTGAAAAACTGATTATTAGAGTCAATTAAGGATAGT
s monDom5.chr8  173163865 36 + 312544902 TTAAGAAACTGGAAATGAGGGTTGAATGACAAACTT
\end{minted}

In each row, the first number indicates the starting position (in zero-based coordinates) of the aligned sequence on the chromosome, followed by the size of the aligned sequence, the strand, the size of the full chromosome, and the aligned sequence.

A \verb+Seq+ object representing such a partially defined sequence can be created using a dictionary for the \verb+data+ argument, where the keys are the starting coordinates of the known sequence segments, and the values are the corresponding sequence contents. For example, for the first sequence we would use

%doctest
\begin{minted}{pycon}
>>> from Bio.Seq import Seq
>>> seq = Seq({117512683: "TTGAAAACCTGAATGTGAGAGTCAGTCAAGGATAGT"}, length=159345973)
\end{minted}

Extracting a subsequence from a partially define sequence may return a fully defined sequence, an undefined sequence, or a partially defined sequence, depending on the coordinates:

%cont-doctest
\begin{minted}{pycon}
>>> seq[1000:1020]
Seq(None, length=20)
>>> seq[117512690:117512700]
Seq('CCTGAATGTG')
>>> seq[117512670:117512690]
Seq({13: 'TTGAAAA'}, length=20)
>>> seq[117512700:]
Seq({0: 'AGAGTCAGTCAAGGATAGT'}, length=41833273)
\end{minted}

Partially defined sequences can also be created by appending sequences, if at least one of the sequences is partially or fully undefined:
%cont-doctest
\begin{minted}{pycon}
>>> seq = Seq("ACGT")
>>> undefined_seq = Seq(None, length=10)
>>> seq + undefined_seq + seq
Seq({0: 'ACGT', 14: 'ACGT'}, length=18)
\end{minted}

\section{MutableSeq objects}
\label{sec:mutable-seq}

Just like the normal Python string, the \verb|Seq| object is ``read only'', or in Python terminology, immutable.  Apart from wanting the \verb|Seq| object to act like a string, this is also a useful default since in many biological applications you want to ensure you are not changing your sequence data:

%doctest
\begin{minted}{pycon}
>>> from Bio.Seq import Seq
>>> my_seq = Seq("GCCATTGTAATGGGCCGCTGAAAGGGTGCCCGA")
\end{minted}

Observe what happens if you try to edit the sequence:
%cont-doctest
\begin{minted}{pycon}
>>> my_seq[5] = "G"
Traceback (most recent call last):
...
TypeError: 'Seq' object does not support item assignment
\end{minted}

However, you can convert it into a mutable sequence (a \verb|MutableSeq| object) and do pretty much anything you want with it:

%cont-doctest
\begin{minted}{pycon}
>>> from Bio.Seq import MutableSeq
>>> mutable_seq = MutableSeq(my_seq)
>>> mutable_seq
MutableSeq('GCCATTGTAATGGGCCGCTGAAAGGGTGCCCGA')
\end{minted}

Alternatively, you can create a \verb|MutableSeq| object directly from a string:

%doctest
\begin{minted}{pycon}
>>> from Bio.Seq import MutableSeq
>>> mutable_seq = MutableSeq("GCCATTGTAATGGGCCGCTGAAAGGGTGCCCGA")
\end{minted}

Either way will give you a sequence object which can be changed:

%cont-doctest
\begin{minted}{pycon}
>>> mutable_seq
MutableSeq('GCCATTGTAATGGGCCGCTGAAAGGGTGCCCGA')
>>> mutable_seq[5] = "C"
>>> mutable_seq
MutableSeq('GCCATCGTAATGGGCCGCTGAAAGGGTGCCCGA')
>>> mutable_seq.remove("T")
>>> mutable_seq
MutableSeq('GCCACGTAATGGGCCGCTGAAAGGGTGCCCGA')
>>> mutable_seq.reverse()
>>> mutable_seq
MutableSeq('AGCCCGTGGGAAAGTCGCCGGGTAATGCACCG')
\end{minted}

Note that the \verb|MutableSeq| object's \verb|reverse()| method, like the \verb|reverse()| method of a Python list, reverses the sequence in place.

An important technical difference between mutable and immutable objects in Python means that you can't use a \verb|MutableSeq| object as a dictionary key, but you can use a Python string or a \verb|Seq| object in this way.

Once you have finished editing your a \verb|MutableSeq| object, it's easy to get back to a read-only \verb|Seq| object should you need to:

%cont-doctest
\begin{minted}{pycon}
>>> from Bio.Seq import Seq
>>> new_seq = Seq(mutable_seq)
>>> new_seq
Seq('AGCCCGTGGGAAAGTCGCCGGGTAATGCACCG')
\end{minted}

You can also get a string from a \verb|MutableSeq| object just like from a \verb|Seq| object (Section~\ref{sec:seq-to-string}).

\section{Working with strings directly}
\label{sec:seq-module-functions}
To close this chapter, for those you who \emph{really} don't want to use the sequence
objects (or who prefer a functional programming style to an object orientated one),
there are module level functions in \verb|Bio.Seq| will accept plain Python strings,
\verb|Seq| objects or \verb|MutableSeq| objects:

%doctest
\begin{minted}{pycon}
>>> from Bio.Seq import reverse_complement, transcribe, back_transcribe, translate
>>> my_string = "GCTGTTATGGGTCGTTGGAAGGGTGGTCGTGCTGCTGGTTAG"
>>> reverse_complement(my_string)
'CTAACCAGCAGCACGACCACCCTTCCAACGACCCATAACAGC'
>>> transcribe(my_string)
'GCUGUUAUGGGUCGUUGGAAGGGUGGUCGUGCUGCUGGUUAG'
>>> back_transcribe(my_string)
'GCTGTTATGGGTCGTTGGAAGGGTGGTCGTGCTGCTGGTTAG'
>>> translate(my_string)
'AVMGRWKGGRAAG*'
\end{minted}

\noindent You are, however, encouraged to work with \verb|Seq| objects by default.
